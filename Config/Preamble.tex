%!TeX spellcheck = en_GB 
% ******************************************************************************
% ****************************** Custom Margin *********************************
% \usepackage{todonotes}
% Add `custommargin' in the document class options to use this section
% Set {innerside margin (use left or inner) / outerside margin (right or outer) / 
% topmargin / bottom margin} and other page dimensions
% Minimum specified in requirements:
\ifsetCustomMargin
  \RequirePackage[inner=35mm,outer=25mm,top=25mm,bottom=25mm,includehead,includefoot]{geometry}
%  \setFancyHdr % To apply fancy header after geometry package is loaded
\fi

% Example with original template
%\ifsetCustomMargin
%  \RequirePackage[left=37mm,right=30mm,top=35mm,bottom=30mm]{geometry}
%  \setFancyHdr % To apply fancy header after geometry package is loaded
%\fi

% Add spaces between paragraphs
\setlength{\parindent}{0pt}
\setlength{\parskip}{\baselineskip}

% Ragged bottom avoids extra whitespaces between paragraphs
\raggedbottom

% To remove the excess top spacing for enumeration, list and description
%\usepackage{enumitem}
%\setlist[enumerate,itemize,description]{topsep=0em}

% Edit spacing around titles
\RequirePackage{titlesec}
\titleformat{\chapter}[display]   
{\normalfont\huge\bfseries}{\chaptertitlename\ \thechapter}{20pt}{\Huge}   
\titlespacing*{\chapter}{0pt}{-40pt}{40pt}
\allowdisplaybreaks
\makeatletter\@openrightfalse

% Edit spacing around figures
\setlength{\abovedisplayskip}{4pt}
\setlength{\belowdisplayskip}{4pt}

% *****************************************************************************
% ******************* Fonts (like different typewriter fonts etc.)*************

% Add `customfont' in the document class option to use this section

\ifsetCustomFont
  % Set your custom font here and use `customfont' in options. Leave empty to
  % load computer modern font (default LaTeX font).
  %\RequirePackage{helvet}

  % For use with XeLaTeX
  %  \setmainfont[
  %    Path              = ./libertine/opentype/,
  %    Extension         = .otf,
  %    UprightFont = LinLibertine_R,
  %    BoldFont = LinLibertine_RZ, % Linux Libertine O Regular Semibold
  %    ItalicFont = LinLibertine_RI,
  %    BoldItalicFont = LinLibertine_RZI, % Linux Libertine O Regular Semibold Italic
  %  ]
  %  {libertine}
  %  % load font from system font
  %  \newfontfamily\libertinesystemfont{Linux Libertine O}
\fi

% ******************************************************************************
% **************************** Custom Packages *********************************

% *************************** Maths and Physics ********************************

\usepackage{mathtools}
\usepackage{amssymb}
\usepackage{amsmath}
\usepackage{amsfonts}
\usepackage{gensymb}
%\usepackage{physics}

% ************************* Algorithms and Pseudocode **************************

\usepackage{algorithm2e}
%\usepackage{algpseudocode}

% ********************Captions and Hyperreferencing / URL **********************

% Captions: This makes captions of figures use a boldfaced small font.
%\RequirePackage[small,bf]{caption}

\RequirePackage[labelsep=space,tableposition=top]{caption}
\renewcommand{\figurename}{Fig.} %to support older versions of captions.sty

\usepackage[nameinlink]{cleveref}
\crefname{app}{Appendix}{Appendices}

% \usepackage[hyphens]{url}
\usepackage{xurl}
\usepackage{hyperref}
\urlstyle{tt}

\usepackage[acronym,nonumberlist,toc]{glossaries}
% \usepackage[acronym]{glossaries}
% \makeglossaries
\makenoidxglossaries

% *************************** Graphics and figures *****************************

%\usepackage{rotating}
%\usepackage{wrapfig}

% Uncomment the following two lines to force Latex to place the figure.
% Use [H] when including graphics. Note 'H' instead of 'h'
\usepackage{float}
\restylefloat{figure}

% Subcaption package is also available in the sty folder you can use that by
% uncommenting the following line
% This is for people stuck with older versions of texlive
%\usepackage{sty/caption/subcaption}
\usepackage{subcaption}

% Support pdf as graphics
\usepackage{graphicx}

\usepackage{tikz}
\usetikzlibrary{shapes,arrows}
\usepackage{geometry}
\usepackage[templates]{genealogytree}
\usepackage{lmodern}

\usepackage{pdfpages}

% ********************************** Tables ************************************
\usepackage{booktabs} % For professional looking tables
\usepackage{multirow}

%\usepackage{multicol}
\usepackage{longtable}
%\usepackage{tabularx}


% *********************************** SI Units *********************************
\usepackage{siunitx} % use this package module for SI units


% ******************************* Line Spacing *********************************

% Choose linespacing as appropriate. Default is one-half line spacing as per the
% University guidelines

% \doublespacing
\onehalfspacing
% \singlespacing


% ************************ Formatting / Footnote *******************************

% Don't break enumeration (etc.) across pages in an ugly manner (default 10000)
%\clubpenalty=500
%\widowpenalty=500

%\usepackage[perpage]{footmisc} % Range of footnote options
%\usepackage[perpage]{footmisc} % Range of footnote options

\usepackage{amsmath}
\usepackage{amssymb}
\usepackage{enumitem}
\SetLabelAlign{myright}{\hss\llap{$#1$}}
\newlist{where}{description}{1}
\setlist[where]{labelwidth=2cm,labelsep=1em,
	leftmargin=!,align=myright,font=\normalfont}

\newcommand{\SubItem}[1]{
    {\setlength\itemindent{15pt} \item[-] #1}
}


% *****************************************************************************
% *************************** Bibliography  and References ********************

%\usepackage{cleveref} %Referencing without need to explicitly state fig /table

% Add `custombib' in the document class option to use this section
\ifuseCustomBib
   \usepackage[numbers]{natbib} % CustomBib

% If you would like to use biblatex for your reference management, as opposed to the default `natbibpackage` pass the option `custombib` in the document class. Comment out the previous line to make sure you don't load the natbib package. Uncomment the following lines and specify the location of references.bib file

%\RequirePackage[backend=biber, style=numeric-comp, citestyle=numeric, sorting=nty, natbib=true]{biblatex}
%\addbibresource{References/references} %Location of references.bib only for biblatex, Do not omit the .bib extension from the filename.

\fi

% changes the default name `Bibliography` -> `References'
\renewcommand{\bibname}{References}


% ******************************************************************************
% ************************* User Defined Commands ******************************

% *********** To change the name of Table of Contents / LOF and LOT ************

%\renewcommand{\contentsname}{My Table of Contents}
%\renewcommand{\listfigurename}{My List of Figures}
%\renewcommand{\listtablename}{My List of Tables}


% ********************** TOC depth and numbering depth *************************

% Victor changed to 3
\setcounter{secnumdepth}{3}
\setcounter{tocdepth}{3}

% ******************************* Nomenclature *********************************

% To change the name of the Nomenclature section, uncomment the following line

%\renewcommand{\nomname}{Symbols}


% ********************************* Appendix ***********************************

% The default value of both \appendixtocname and \appendixpagename is 
% `Appendices'. These names can all be changed via:

%\renewcommand{\appendixtocname}{List of appendices}
%\renewcommand{\appendixname}{Appndx}
\usepackage[titletoc]{appendix}

% *********************** Configure Draft Mode **********************************

% Uncomment to disable figures in `draft'
%\setkeys{Gin}{draft=true}  % set draft to false to enable figures in `draft'

% These options are active only during the draft mode
% Default text is "Draft"
\SetDraftText{Thesis Draft v0.1}

% Default Watermark location is top. Location (top/bottom)
%\SetDraftWMPosition{bottom}

% Draft Version - default is v1.0
\SetDraftVersion{GENERAL}

% Draft Text grayscale value (should be between 0-black and 1-white)
% Default value is 0.75
%\SetDraftGrayScale{0.8}


% ******************************** Todo Notes **********************************
% Uncomment the following lines to have todonotes.
% You can have multiple notes (uncomment \othernote line), e.g. to write notes 
% directed at your supervisor.

\makeatletter
\ifsetDraft
	\usepackage[colorinlistoftodos]{todonotes}
	\newcommand{\oftodo}[1]{\todo[size=\small,author=@OF,inline,color=orange!40]{#1}}
	\newcommand{\swtodo}[1]{\todo[size=\small,inline,author=@SW,color=red!40]{#1}}
	\newcommand{\actodo}[1]{\todo[size=\small,inline,author=@Accn,color=magenta!40]{#1}}
\else
	\newcommand{\oftodo}[1]{}
	\newcommand{\swtodo}[1]{}
	\newcommand{\actodo}[1]{}
	% \newcommand{\listoftodos}{}
	\providecommand{\@LN}[2]{}
\fi
\makeatother

% Example todo: \mynote{Hey! I have a note}

% *************************** Other custom commands ****************************

% At numbers in a long align* block
\newcommand{\numberthis}{\addtocounter{equation}{1}\tag{\theequation}}

% Custom maths symbols (examples)
%\newcommand{\Rwb}{\prescript{W}{}{R}_B}
%\newcommand{\dRwb}{\prescript{W}{}{\dot{R}}_B}
%\newcommand{\skewm}[3]{\bmqty{  0 & -#3 &  #2 \\
%							   #3 &   0 & -#1 \\
%						      -#2 &  #1 &   0 }}
%\DeclareMathOperator*{\argmin}{arg\,min}

% Victor changes
\usepackage{dirtytalk}
\newtheorem{definition}{Definition}


% Python highlighting https://tex.stackexchange.com/questions/83882/how-to-highlight-python-syntax-in-latex-listings-lstinputlistings-command

% Default fixed font does not support bold face
\DeclareFixedFont{\ttb}{T1}{txtt}{bx}{n}{10} % for bold
\DeclareFixedFont{\ttm}{T1}{txtt}{m}{n}{10}  % for normal

% Custom colors
\usepackage{color}
\definecolor{deepblue}{rgb}{0,0,0.5}
\definecolor{deepred}{rgb}{0.6,0,0}
\definecolor{deepgreen}{rgb}{0,0.5,0}
\definecolor{codegray}{rgb}{0.5,0.5,0.5}

% \usepackage{todonotes}
\usepackage{listings}

% Python style for highlighting
\newcommand\pythonstyle{\lstset{
language=Python,
basicstyle=\ttm,
morekeywords={self},              % Add keywords here
keywordstyle=\ttb\color{deepblue},
numberstyle=\tiny\color{codegray},
numbers=left,
emph={MyClass,__init__},          % Custom highlighting
emphstyle=\ttb\color{deepred},    % Custom highlighting style
stringstyle=\color{deepgreen},
frame=tb,                         % Any extra options here
showstringspaces=false,
breaklines=true,
postbreak=\mbox{\textcolor{red}{$\hookrightarrow$}\space},
commentstyle=\color{gray}\ttfamily
}}


% Python environment
\lstnewenvironment{python}[1][]
{
\pythonstyle
\lstset{#1}
}
{}

% Python for external files
\newcommand\pythonexternal[2][]{{
\pythonstyle
\lstinputlisting[#1]{#2}}}

% Python for inline
\newcommand\pythoninline[1]{{\pythonstyle\lstinline!#1!}}

% For centering large tables
\usepackage{adjustbox}

% Inline code snippet
\newcommand{\code}[1]{\texttt{#1}}
% \newcommand\myworries[1]{\textcolor{red}{#1}}
\usepackage{todonotes}