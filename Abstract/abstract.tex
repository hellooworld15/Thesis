\cleardoublepage
\begin{abstract}
 Micro Aerial Vehicles (MAVs) are a class of unmanned aerial vehicles that vary from the size of an insect to sizes similar to large birds. They are difficult to design and manufacture, due to several main factors, including low-speed flight, small physical dimensions, structural strength, reduced stall speed and low inertia. With MAV design becoming one of the fastest-growing areas of development in the aerospace industry, there is a need for experimental data to model varieties of MAV configurations. This investigation outlines propellers' influence on MAVs' aerodynamics and stability. \\
 The aerodynamic analysis of the MAV showed that the addition of the propeller delayed the MAVs stall in all configurations. This effect is seen to a greater extent for the tractor configuration. The tractor configuration also had a deeper stall when airspeed is at 20m$s^{-1}$, which is undesirable to recover the MAV if a stall were to occur. The tractor configuration, in general, produced less drag than the pusher configuration.
 The MAV's stability analysis showed that the MAV model's pitching moment decreased as the angle of attack increased and indicated that the MAV was stable in all cases. The pusher configuration had a larger pitching moment compared with the no-propeller model, hence MAV became more stable in the pusher configuration. The tractor configuration showed the opposite, and the MAV became less stable in this configuration. The pusher configuration shifts the rolling moment to a more significant extent than the tractor configuration as it has a longer moment arm from the position of the propeller to the aerodynamic centre of the MAV model. Further investigation is required into the suitability of VAP 3.5 as a validation software for wing propeller interactions. Currently, the side forces and moment calculations are still in testing and based on discrepancies found with wind tunnel testing, further work is required. 
    

\end{abstract}
