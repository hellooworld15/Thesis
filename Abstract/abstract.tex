\cleardoublepage
\begin{abstract}
 Micro Aerial Vehicles (MAVs) are a class of unmanned aerial vehicles that are much smaller in size. These small aircraft allow for the observation of hazardous environments which are inaccessible to ground vehicles. Due to their small size they are much more difficult to design and manufacture. The additional complexity is due to several main factors including: low speed flight, small physical dimensions, structural strength, reduced stall speed and low inertia. With MAV design becoming one of the fastest-growing areas of development in the aerospace industry, there is an increasing need to have accurate experimental data to validate, simulate and model varieties of MAV configurations. This investigation outlines the influence of propellers on the aerodynamics and stability of MAV's. \\
 The aerodynamic analysis of the MAV showed that the addition of the propeller delayed the MAVs stall in all configurations, however this effect is seen to a much greater extent for the tractor configuration. The tractor configuration did also have a deeper stall when airspeed is at 20m$s^{-1}$ which is undesirable in order to be able to recover the MAV if a stall were to occur. The highest drag coefficient occurred when no propeller operated on the MAV model. The tractor configuration, in general, produces less drag than the pusher configuration; hence, the drag coefficient is lower for the tractor configuration. The addition of the propeller increased thrust and hence lowered the coefficient of drag for the pusher and tractor configurations. \\
 The stability analysis of the MAV showed that the pitching moment of the MAV model decreased as the angle of attack increased and indicated that the MAV was stable in all cases. The pusher configuration had a larger pitching moment compared with the no-propeller model and showed that the MAV became more stable in the pusher configuration. The tractor configuration showed the opposite and the MAV became less stable in this configuration. The pusher configuration shifts the rolling moment to a larger extent due to having a larger moment arm from the position of the propeller to the aerodynamic centre of the MAV model than the tractor configuration. Further investigation is required into the suitability of VAP 3.5 as a validation software for wing propeller interactions. Currently the side forces and moment calculations are still in testing and based on discrepancies found with wind tunnel testing, further work is required. 
    

\end{abstract}
